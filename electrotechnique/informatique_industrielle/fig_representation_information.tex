\documentclass[a4paper, 11pt, twoside, fleqn]{memoir}

\usepackage{AOCDTF}


\typemedia{paper} %choix screen ou paper pour les vidéos et schémas animés

\decoupagechapitre{1} %juste pour éviter les erreurs lors de la compilation des sous-programmations (passera en commentaire)
\marqueurchapitre

%lien d'édition des figures Tikz sur le site mathcha.io (rajouter le lien d'une modification effectuée sur la figure tikz avec le nom du modificateur car il n'y a qu'un lien par compte)

%lien mathcha Nom Prénom : https://www.mathcha.io/editor/pVWMLi4YiMxhw5uKW8d4df07dDyjIdG9DDQFKDKJ5D

%--------------------------------------
%corps du document
%--------------------------------------

\begin{document} %corps du document

	%--------------------------------------
	%espace de rédaction du document
	%--------------------------------------
	



\tikzset{every picture/.style={line width=0.75pt}} %set default line width to 0.75pt        

\begin{tikzpicture}[x=0.75pt,y=0.75pt,yscale=-1,xscale=1]
%uncomment if require: \path (0,511); %set diagram left start at 0, and has height of 511

%Shape: Rectangle [id:dp924787478324178] 
\draw   (110,10) -- (210,10) -- (210,50) -- (110,50) -- cycle ;
%Shape: Rectangle [id:dp7018483899820586] 
\draw   (10,90) -- (110,90) -- (110,130) -- (10,130) -- cycle ;
%Shape: Rectangle [id:dp47642399123490575] 
\draw   (210,90) -- (310,90) -- (310,130) -- (210,130) -- cycle ;
%Shape: Rectangle [id:dp07730451607913758] 
\draw   (110,170) -- (210,170) -- (210,210) -- (110,210) -- cycle ;
%Shape: Rectangle [id:dp5358779976921059] 
\draw   (310,170) -- (410,170) -- (410,210) -- (310,210) -- cycle ;
%Shape: Rectangle [id:dp5383180839088969] 
\draw   (210,250) -- (310,250) -- (310,290) -- (210,290) -- cycle ;
%Shape: Rectangle [id:dp9142273030376454] 
\draw   (410,250) -- (510,250) -- (510,290) -- (410,290) -- cycle ;
%Shape: Rectangle [id:dp5301195417698047] 
\draw   (310,330) -- (410,330) -- (410,370) -- (310,370) -- cycle ;
%Shape: Rectangle [id:dp42705394536796926] 
\draw   (310,410) -- (410,410) -- (410,450) -- (310,450) -- cycle ;
%Shape: Rectangle [id:dp16659373317524484] 
\draw   (510,330) -- (610,330) -- (610,370) -- (510,370) -- cycle ;
%Shape: Rectangle [id:dp8438434637646167] 
\draw   (510,410) -- (610,410) -- (610,450) -- (510,450) -- cycle ;
%Straight Lines [id:da08524848516394623] 
\draw    (160,50) -- (160,70) -- (60,70) -- (60,90) ;
%Straight Lines [id:da8352257615142615] 
\draw    (260,90) -- (260,70) -- (160,70) ;
%Straight Lines [id:da5488912619407663] 
\draw    (260,130) -- (260,150) -- (160,150) -- (160,170) ;
%Straight Lines [id:da6384915344399713] 
\draw    (360,170) -- (360,150) -- (260,150) ;
%Straight Lines [id:da6407872178965494] 
\draw    (360,210) -- (360,230) -- (260,230) -- (260,250) ;
%Straight Lines [id:da7195357364662622] 
\draw    (460,250) -- (460,230) -- (360,230) ;
%Straight Lines [id:da7322405590008245] 
\draw    (260,290) -- (260,310) -- (260,430) -- (290,430) -- (310,430) ;
%Straight Lines [id:da5239926046301219] 
\draw    (510,430) -- (460,430) -- (460,290) ;
%Straight Lines [id:da9381297642177667] 
\draw    (460,350) -- (510,350) ;
%Straight Lines [id:da13925475186742386] 
\draw    (260,350) -- (310,350) ;

% Text Node
\draw (160,30) node   [align=left] {Information};
% Text Node
\draw (60,110) node   [align=left] {Instructions};
% Text Node
\draw (260,110) node   [align=left] {Données};
% Text Node
\draw (160,190) node   [align=left] {Caractère};
% Text Node
\draw (360,190) node   [align=left] {Numérique};
% Text Node
\draw (260,270) node   [align=left] {Réel};
% Text Node
\draw (460,270) node   [align=left] {Entier};
% Text Node
\draw (360,350) node   [align=left] {Virgule\\fixe};
% Text Node
\draw (360,430) node   [align=left] {Virgule\\flottante};
% Text Node
\draw (560,350) node   [align=left] {Non signé};
% Text Node
\draw (560,430) node   [align=left] {Signé};


\end{tikzpicture}


	
	
	
\end{document}


