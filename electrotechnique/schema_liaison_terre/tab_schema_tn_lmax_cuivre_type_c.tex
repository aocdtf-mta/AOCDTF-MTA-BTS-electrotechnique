%--------------------------------------
%ELECTROTECHNIQUE - SCHEMA DE LIAISON A LA TERRE
%--------------------------------------

%utiliser les environnement \begin{comment} \end{comment} pour mettre en commentaire le préambule une fois la programmation appelée dans le document maître (!ne pas oublier de mettre en commentaire \end{document}!)

\begin{comment}

\documentclass[a4paper, 11pt, twoside, fleqn]{memoir}

\usepackage{AOCDTF}

\marqueurchapitre

%lien d'édition des figures Tikz sur le site mathcha.io (rajouter le lien d'une modification effectuée sur la figure tikz avec le nom du modificateur car il n'y a qu'un lien par compte)

%lien mathcha Nom Prénom : 

%--------------------------------------
%corps du document
%--------------------------------------

\begin{document} %corps du document
	\openleft %début de chapitre à gauche

\end{comment}

\begin{xltabular}{\linewidth}{c CCCCCCCCCCCCCCCC}
\caption{$L_{max}$ des circuits en mètre selon les sections des conducteurs en cuivre en schéma TN pour les disjoncteurs domestiques de type C\supercite{Schneider:schematncalculdefaut}} \\
\toprule
\multirow[c]{2}{*}{\thead{Section des\\conducteurs\\(\si{\square\milli\meter})}}	& \multicolumn{16}{l}{\thead{Courant assigné (\si{\ampere})}} \\
\cmidrule(lr){2-17} 
& \mcrot{1}{l}{60}{1} 	& \mcrot{1}{l}{60}{2} & \mcrot{1}{l}{60}{3}	& \mcrot{1}{l}{60}{4} & \mcrot{1}{l}{60}{6} & \mcrot{1}{l}{60}{10} & \mcrot{1}{l}{60}{16} & \mcrot{1}{l}{60}{20} &\mcrot{1}{l}{60}{25} & \mcrot{1}{l}{60}{32} & \mcrot{1}{l}{60}{40} & \mcrot{1}{l}{60}{50} & \mcrot{1}{l}{60}{63} & \mcrot{1}{l}{60}{80} & \mcrot{1}{l}{60}{100} & \mcrot{1}{l}{60}{125} \\
\midrule
\endfirsthead
\multicolumn{17}{l}{\small\textit{Page précédente}} \\
\midrule
\multirow[c]{2}{*}{\thead{Section des\\conducteurs\\(\si{\square\milli\meter})}}	& \multicolumn{16}{l}{\thead{Courant assigné (\si{\ampere})}} \\
\cmidrule(lr){2-17} 
& \mcrot{1}{l}{60}{1} 	& \mcrot{1}{l}{60}{2} & \mcrot{1}{l}{60}{3}	& \mcrot{1}{l}{60}{4} & \mcrot{1}{l}{60}{6} & \mcrot{1}{l}{60}{10} & \mcrot{1}{l}{60}{16} & \mcrot{1}{l}{60}{20} &\mcrot{1}{l}{60}{25} & \mcrot{1}{l}{60}{32} & \mcrot{1}{l}{60}{40} & \mcrot{1}{l}{60}{50} & \mcrot{1}{l}{60}{63} & \mcrot{1}{l}{60}{80} & \mcrot{1}{l}{60}{100} & \mcrot{1}{l}{60}{125} \\
\midrule
\endhead
\midrule %filet de milieu de tableau
\multicolumn{17}{r}{\small\textit{Page suivante}}
\endfoot
\bottomrule
\endlastfoot %pied de page de toutes les pages du tableau
1,5		&	600	&	300	&	200	&	150	&	100	&	60		&	37		&	30		&	24		&	18		&	15		&	12	 	&	9		&	7		&	6		&	5 \\
2,5		&			&	500	&	333	&	250	&	167	&	100	&	62		&	50		&	40		&	31		&	25		&	20		&	16		&	12		&	10		&	8 \\
4			&			&			&	533	&	400	&	267	&	160	&	100	&	80		&	64		&	50		&	40		&	32		&	25		&	20		&	16		&	13 \\
6			&			&			&			&	600	&	400	&	240	&	150	&	120	&	96		&	75		&	60		&	48		&	38		&	30		&	24		&	19 \\
10			&			&			&			&			&	677	&	400	&	250	&	200	&	160	&	125	&	100	&	80		&	63		&	50		&	40		&	32 \\
16			&			&			&			&			&			&	640	&	400	&	320	&	256	&	200	&	160	&	128	&	101	&	80		&	64		&	51 \\
25			&			&			&			&			&			&			&	625	&	500	&	400	&	312	&	250	&	200	&	159	&	125	&	100	&	80 \\
35			&			&			&			&			&			&			&	875	&	700	&	560	&	437	&	350	&	280	&	222	&	175	&	140	&	112 \\
50			&			&			&			&			&			&			&			&			&	760	&	594	&	475	&	380	&	301	&	237	&	190	&	152 \\
\end{xltabular}



%\end{document}

