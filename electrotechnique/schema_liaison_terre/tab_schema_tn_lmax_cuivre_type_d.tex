%--------------------------------------
%ELECTROTECHNIQUE - SCHEMA DE LIAISON A LA TERRE
%--------------------------------------

%utiliser les environnement \begin{comment} \end{comment} pour mettre en commentaire le préambule une fois la programmation appelée dans le document maître (!ne pas oublier de mettre en commentaire \end{document}!)


\documentclass[a4paper, 11pt, twoside, fleqn]{memoir}

\usepackage{AOCDTF}

\marqueurchapitre
\decoupagechapitre{1} %juste pour éviter les erreurs lors de la compilation des sous-programmations (passera en commentaire)

%lien d'édition des figures Tikz sur le site mathcha.io (rajouter le lien d'une modification effectuée sur la figure tikz avec le nom du modificateur car il n'y a qu'un lien par compte)

%lien mathcha Nom Prénom : 

%--------------------------------------
%corps du document
%--------------------------------------

\begin{document} %corps du document
	\openleft %début de chapitre à gauche


\begin{xltabular}{\linewidth}{c CCCCCCCCCCCCCCCC}
\caption{$L_{max}$ des circuits en mètre selon les sections des conducteurs en cuivre en schéma TN pour les disjoncteurs domestiques de type D\supercite{Schneider:schematncalculdefaut}} \\
\toprule
\multirow[c]{2}{*}{\thead{Section des\\conducteurs\\(\si{\square\milli\meter})}}	& \multicolumn{16}{l}{\thead{Courant assigné (\si{\ampere})}} \\
\cmidrule(lr){2-17} 
& \mcrot{1}{l}{60}{1} 	& \mcrot{1}{l}{60}{2} & \mcrot{1}{l}{60}{3}	& \mcrot{1}{l}{60}{4} & \mcrot{1}{l}{60}{6} & \mcrot{1}{l}{60}{10} & \mcrot{1}{l}{60}{16} & \mcrot{1}{l}{60}{20} &\mcrot{1}{l}{60}{25} & \mcrot{1}{l}{60}{32} & \mcrot{1}{l}{60}{40} & \mcrot{1}{l}{60}{50} & \mcrot{1}{l}{60}{63} & \mcrot{1}{l}{60}{80} & \mcrot{1}{l}{60}{100} & \mcrot{1}{l}{60}{125} \\
\midrule
\endfirsthead
\multicolumn{17}{l}{\small\textit{Page précédente}} \\
\midrule
\multirow[c]{2}{*}{\thead{Section des\\conducteurs\\(\si{\square\milli\meter})}}	& \multicolumn{16}{l}{\thead{Courant assigné (\si{\ampere})}} \\
\cmidrule(lr){2-17} 
& \mcrot{1}{l}{60}{1} 	& \mcrot{1}{l}{60}{2} & \mcrot{1}{l}{60}{3}	& \mcrot{1}{l}{60}{4} & \mcrot{1}{l}{60}{6} & \mcrot{1}{l}{60}{10} & \mcrot{1}{l}{60}{16} & \mcrot{1}{l}{60}{20} &\mcrot{1}{l}{60}{25} & \mcrot{1}{l}{60}{32} & \mcrot{1}{l}{60}{40} & \mcrot{1}{l}{60}{50} & \mcrot{1}{l}{60}{63} & \mcrot{1}{l}{60}{80} & \mcrot{1}{l}{60}{100} & \mcrot{1}{l}{60}{125} \\
\midrule
\endhead
\midrule %filet de milieu de tableau
\multicolumn{17}{r}{\small\textit{Page suivante}}
\endfoot
\bottomrule
\endlastfoot %pied de page de toutes les pages du tableau
1,5		&	429		&	214		&	143		&	107		&	71		&	43		&	27		&	21		&	17		&	13		&	11		&	9		&	7		&	5		&	4		&	3 \\
\middashrule
2,5		&	714		&	357		&	238		&	179		&	119	&	71		&	45		&	36		&	29		&	22		&	18		&	14		&	11		&	9		&	7		&	6 \\
\middashrule
4			&				&	571		&	381		&	286		&	190	&	114	&	71		&	57		&	46		&	36		&	29		&	23		&	18		&	14		&	11		&	9 \\
\middashrule
6			&				&	857		&	571		&	429		&	286	&	171	&	107	&	86		&	69		&	54		&	43		&	34		&	27		&	21		&	17		&	14 \\
\middashrule
10			&				&				&	952		&	714		&	476	&	286	&	179	&	143	&	114	&	89		&	71		&	57		&	45		&	36		&	29		&	23 \\
\middashrule
16			&				&				&				&				&	762	&	457	&	286	&	229	&	183	&	143	&	114	&	91		&	73		&	57		&	46		&	37 \\
\middashrule
25			&				&				&				&				&			&	714	&	446	&	357	&	286	&	223	&	179	&	143	&	113	&	89		&	71		&	57 \\
\middashrule
35			&				&				&				&				&			&			&	625	&	500	&	400	&	313	&	250	&	200	&	159	&	125	&	100	&	80 \\
\middashrule
50			&				&				&				&				&			&			&			&	679	&	543	&	424	&	339	&	271	&	215	&	170	&	136	&	109 \\
\end{xltabular}



\end{document}

