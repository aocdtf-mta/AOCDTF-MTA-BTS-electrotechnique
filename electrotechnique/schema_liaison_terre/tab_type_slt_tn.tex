%--------------------------------------
%ELECTROTECHNIQUE - SCHEMA DE LIAISON A LA TERRE
%--------------------------------------

%utiliser les environnement \begin{comment} \end{comment} pour mettre en commentaire le préambule une fois la programmation appelée dans le document maître (!ne pas oublier de mettre en commentaire \end{document}!)

\begin{comment}

\documentclass[a4paper, 11pt, twoside, fleqn]{memoir}

\usepackage{AOCDTF}

\marqueurchapitre

%lien d'édition des figures Tikz sur le site mathcha.io (rajouter le lien d'une modification effectuée sur la figure tikz avec le nom du modificateur car il n'y a qu'un lien par compte)

%lien mathcha Nom Prénom : 

%--------------------------------------
%corps du document
%--------------------------------------

\begin{document} %corps du document
	\openleft %début de chapitre à gauche

\end{comment}

\begin{xltabular}{\textwidth}{l >{\compress}X >{\compress}X >{\compress}X}
\caption{Déclinaisons du SLT TN}\\
\toprule
	& \thead{Caractéristiques}		& \thead{Avantages}		& \thead{Inconvénients} \\
\midrule
\endfirsthead %en-tête de la première page du tableau  
\multicolumn{4}{l}{\small\textit{Page précédente}} \\
\midrule %filet de milieu de tableau
 	&	\thead{Caractéristique}		& \thead{Avantages} 	& \thead{Inconvénients} \\
\midrule
\endhead
\midrule %filet de milieu de tableau
\multicolumn{4}{r}{\small\textit{Page suivante}} \\
\endfoot %pied de page de toutes les pages du tableau
\bottomrule
\endlastfoot %pied de page de la dernièredu tableau
\addlinespace
\multicolumn{4}{l}{\textit{Confondus (TN-C)}}	\\ 
\middashrule
&
\begin{tabitemize}
\item conducteurs neutre et PE confondus\,;
\item PE et neutre vert/jaune nommé conducteur Protection \'Equipotentielle Neutre (PEN).
\end{tabitemize}
&
\begin{tabitemize}
\item économie d'un câble.
\end{tabitemize}
&		
\begin{tabitemize}
\item utilisation de canalisations fixes et rigides\,;
\item interdiction de pose :
	\begin{compactitemize}
	\item locaux à risques d'incendies\,;
	\item alimentation d'équipements de traitement de l'information (présence de courant harmonique dans le neutre).
	\end{compactitemize}
\end{tabitemize}\\
\addlinespace
\multicolumn{4}{l}{\textit{Séparés (TN-S)}}	\\ 
\middashrule
&
\begin{tabitemize}
\item conducteurs neutre et PE séparés\,;
\item PE et neutre vert/jaune séparés (PE+N).
\end{tabitemize}
&
\begin{tabitemize}
\item usage de conducteurs souples autorisés\,;
\item séparation et protection du neutre possibles dans les locaux pollués.
\end{tabitemize}
&
\begin{tabitemize}
\item solution plus coûteuse que le schéma TN-C.
\end{tabitemize}
\\
\addlinespace
\multicolumn{4}{l}{\textit{Mixte (TN-C-S)}}	\\ 
\middashrule	
&
\begin{tabitemize}
\item combinaison des SLT TN-C et TN-S dans une même installation\,;
\item usage du SLT TN-C formellement interdit en aval du SLT TN-S.
\end{tabitemize}
&
\begin{tabitemize}
\item combinaison des avantages des deux SLT TN.
\end{tabitemize}
&\\
\end{xltabular}


%\end{document}

