%--------------------------------------
%ELECTROTECHNIQUE - SCHEMA DE LIAISON A LA TERRE
%--------------------------------------

%utiliser les environnement \begin{comment} \end{comment} pour mettre en commentaire le préambule une fois la programmation appelée dans le document maître (!ne pas oublier de mettre en commentaire \end{document}!)

\begin{comment}

\documentclass[a4paper, 11pt, twoside, fleqn]{memoir}

\usepackage{AOCDTF}

\marqueurchapitre
\decoupagechapitre{1} %juste pour éviter les erreurs lors de la compilation des sous-programmations (passera en commentaire)

%--------------------------------------
%corps du document
%--------------------------------------

\begin{document} %corps du document
	\openleft %début de chapitre à gauche

\end{comment}

\chapter{Choix d'un schéma de liaison à la terre}
\ChapFrame

\section{Introduction}

Les différents schémas de liaison à la terre présentent chacun des avantages et des inconvénients, ils sont recommandés selon les critères suivants pour le choix d'un SLT ou de plusieurs SLT imbriqués les uns dans les autres :

\begin{itemize}
\item lois et décrets\,;
\item protection des personnes contre les chocs électriques\,;
\item protection des biens contre les incendies ou explosions d'origine électrique\,;
\item continuité de service\,;
\item protection contre les surtensions\,;
\item compatibilité électromagnétique\,;
\item coût de revient de l'installation.
\end{itemize} 

\section{Lois et décrets}

Le choix d'un SLT est parfois fortement recommandé voire imposé par la législation en vigueur :

%--------------------------------------
%ELECTROTECHNIQUE - SCHEMA DE LIAISON A LA TERRE
%--------------------------------------

%utiliser les environnement \begin{comment} \end{comment} pour mettre en commentaire le préambule une fois la programmation appelée dans le document maître (!ne pas oublier de mettre en commentaire \end{document}!)

\begin{comment}

\documentclass[a4paper, 11pt, twoside, fleqn]{memoir}

\usepackage{AOCDTF}

\marqueurchapitre
\decoupagechapitre{1} %juste pour éviter les erreurs lors de la compilation des sous-programmations (passera en commentaire)

%lien d'édition des figures Tikz sur le site mathcha.io (rajouter le lien d'une modification effectuée sur la figure tikz avec le nom du modificateur car il n'y a qu'un lien par compte)

%lien éditeur Bruno Douchy : https://www.mathcha.io/editor/zjygnFElSdyhJ72e3zT5ZgqwBT4DKnovswpXn1q

%--------------------------------------
%corps du document
%--------------------------------------

\begin{document} %corps du document
	\openleft %début de chapitre à gauche

\end{comment}

\begin{table}[H]
\caption{Législation encadrant le choix d'un SLT}
\begin{tabularx}{\textwidth}{l X X}
\toprule
 & \thead{Utilisation} & \thead{Textes de lois} \\
\midrule
\multicolumn{3}{l}{\textit{Neutre à la terre (TT)}} \\
\middashrule
	&	Bâtiment alimenté par un réseaux de distribution publique (habitat, petit tertiaire, petit atelier, commerce\ldots)	 & Arrêté interministériel du 13/02/1970 \\
\addlinespace
\multicolumn{3}{l}{\textit{Neutre isolé (IT)}} \\
\middashrule
	&	\'Etablissement Recevant du Public (ERP) 	& Règlement de sécurité contre les risques de panique et d'incendie dans les ERP \\
\addlinespace
\multicolumn{3}{l}{\textit{Neutre isolé (TT)}} \\
\middashrule
	&	Circuits d'éclairage de sécurité soumis au décret de protection des travailleurs & Arrêté interministériel du 10/11/1976 relatifs aux circuits et installations de sécurité (J.O. \no{}102 NC du 01/12/1976) \\
\addlinespace
\multicolumn{3}{l}{\textit{Neutre isolé (IT) ou neutre à la terre (TT)}} \\
\middashrule
	&	Mines et carrières & Décret \no{}76-48 du 09/01/1976, circulaire du 09/01/1976 et règlement sur la protection du personnel dans les mines et carrières, annexée au décret 76-48 \\
\bottomrule 
\end{tabularx}
\end{table}

%\end{document}



\section{Protection des personnes contre les chocs électriques}

Pour ce critère, les trois SLT assurent une protection des personnes considérée comme équivalente si les principes d'installation sont bien respectés. Toutefois, le SLT TN exige des compétences techniques en électricité lors des calculs des impédances de boucles de court-circuit à l'installation mais également lors d'extensions de l'installation. Il conviendra d'être vigilant lors des installations de ces extensions et spécialement pour la re-calibration des protections.

\section{Protection des biens contre les incendies ou explosions d'origine électrique}

De part l'installation de DDR, une exploitation correcte des installations en schéma IT et TT conduit à un risque d'incendie quasi-nul. Le SLT IT est même recommandé dans les installations à fort risque explosif. Pour autant, le SLT TN-C présente un risque d'incendie plus élevé. 

\section{Continuité de service}

La continuité de service caractérise l'aptitude d'une installation électrique à assurer un fonctionnement le plus longtemps possible sans coupure. Cette caractéristique est primordiale dans les installations dites \emph{sensibles} ou la sécurité des personnes est en jeu (médical, militaire, éclairage de secours\ldots) ou dans les installations dont les arrêts peuvent engendrer des pertes financières importantes (ligne de production, événementiel\ldots).\\
Dans ces cas-là, le SLT IT est le choix de prédilection parce qu'il permet cette continuité de service lors d'un premier défaut d'isolement. Toutefois, du fait de la propension naturelle des installations à accumuler les défauts d'isolement avec l'âge et les conditions, et du fait de l'obligation d'avoir une équipe de maintenance qualifiée et disponible pour prospecter au premier défaut, on se tourne vers d'autres solutions techniques permettant d'assurer une continuité de service (multiplication des sources principales et de secours\ldots). 

\section{Protection contre les surtensions}

Une surtension peut apparaître sur l'installation basse tension lors d'un claquage sur la partie HT de l'installation, ou plus fréquemment en raison de la foudre. Lorsque celle-ci frappe le sol, le potentiel des prises de terre va s'élever de manière significative à proximité de l'impact et mettre à mal l'équipotentialité des masses conductrices.\\
Pour palier à cette problématique, et sur tous les SLT dans les zones à haut niveau kéraunique AQ2 (classification de densité d'impact de foudre), il est nécessaire d'installer un parafoudre.\\
En schéma TT et TN-S, il doivent être installés en \emph{mode commun} et en \emph{mode différentiel} (un parafoudre au plus proche de chaque équipement). En schéma IT et TN-C, ils ne doivent installés qu'en \emph{mode commun}.

\section{Compatibilité électromagnétique}

Les appareils électriques de type \emph{courants faibles} (informatique, électronique\ldots) sont sensibles aux perturbations électromagnétiques engendrés par le passage du \emph{courant fort} dans les conducteurs à proximité. Le schéma TN-C provoquant des courants de court-circuit à chaque défaut d'isolement, il est fortement déconseillé d'alimenter des appareils sensibles sous ce schéma.

\section{Le coût de revient}

Ce critère est décisif dans le choix d'un SLT car les trois SLT ne sont pas équivalents d'un point de vue économique. Différents coûts sont à prendre en compte lors de la conception (calculs), de l'installation (prix du matériel spécifique) et d'exploitation (entretien par un personnel qualifié ou non). Le moins onéreux sera le SLT TN, suivi du TT et le SLT IT sera le plus coûteux.

\section{Tableau récapitulatif des différents schémas de liaison à la terre}

%--------------------------------------
%ELECTROTECHNIQUE - SCHEMA DE LIAISON A LA TERRE
%--------------------------------------

%utiliser les environnement \begin{comment} \end{comment} pour mettre en commentaire le préambule une fois la programmation appelée dans le document maître (!ne pas oublier de mettre en commentaire \end{document}!)

\begin{comment}

\documentclass[a4paper, 11pt, twoside, fleqn]{memoir}

\usepackage{AOCDTF}

\marqueurchapitre
\decoupagechapitre{1} %juste pour éviter les erreurs lors de la compilation des sous-programmations (passera en commentaire)

%lien d'édition des figures Tikz sur le site mathcha.io (rajouter le lien d'une modification effectuée sur la figure tikz avec le nom du modificateur car il n'y a qu'un lien par compte)

%lien éditeur Bruno Douchy : https://www.mathcha.io/editor/zjygnFElSdyhJ72e3zT5ZgqwBT4DKnovswpXn1q

%--------------------------------------
%corps du document
%--------------------------------------

\begin{document} %corps du document
	\openleft %début de chapitre à gauche

\end{comment}

\begin{landscape}
\begin{xltabular}{\linewidth}{X l c c c c c}
\caption{Comparaison des différents schémas de liaison à la terre} \\
\toprule
\multicolumn{2}{c}{\thead{Critères de comparaison}} & \thead{TT} & \thead{TN-S} & \thead{TN-C} & \thead{IT\\individuelles} & \thead{IT\\interconnectées} \\
\midrule
\endfirsthead
\multicolumn{7}{l}{\small\textit{Page précédente}} \\
\midrule
\multicolumn{2}{c}{\thead{Critères de comparaison}} & \thead{TT} & \thead{TN-S} & \thead{TN-C} & \thead{IT\\individuelles} & \thead{IT\\interconnectées} \\
\midrule
\endhead
\midrule
\multicolumn{7}{r}{\small\textit{Page suivante}} \\
\endfoot
\bottomrule
\endlastfoot
Protection des personnes contre les chocs électriques 														& contacts directs							& + 	& + 	& + 	& + 	& + \\
																																	&	contacts indirects						& + 	& + 	& + 	& + 	& + \\
\addlinespace		
Protection des biens contre les risques d'incendie ou d'explosion d'origine électrique 		& incendie et explosion					&  - 	& - -  	& interdit 	& + 	& - - \\
\addlinespace		
Continuité de service							& creux de tension 				& + & - & - & ++ & - \\
														& sélectivité							& - & + & + & ++ & + \\
														& déclenchement					& - & - & - & + & - \\
														& temps de recherche			& - & + & + & - & + \\
														& temps de réparation			& - - & - - - & - - - & - & - - - \\
\addlinespace		
Protection contre les surtensions		& foudre sur la HT					& - & + & + & + & + \\
														& claquage du transformateur	& - & + & + & + & + \\
\addlinespace
Compatibilité électromagnétique 		& rayonnements 					& + & - & - - & ++ & - \\
														& chute de tension					&  + & - & - & ++ & - \\
														& harmoniques						&  + & + & - - & + & + \\
\addlinespace
Coût à la conception					 		& étude la sélectivité 			& - & + & + & ++ & + \\
														& calcul de $L_{max}$			& + & - & - & ++ & - \\
\addlinespace
Coût à l'installation					 		& nombre de câbles	 			& + & + & ++ & + & + \\
														& nombre de pôles		 			& + & + & ++ & + & + \\
														& pose des câbles		 			& - & - - & - - & ++ & - - \\
														& matériel spécifiques	 		& - & + & + & - & + \\
\addlinespace
Coût à l'exploitation							& recherche de défauts			& - & + & + & - - & + \\
														& coûts des réparations			& - - & - - - & - - - & - &  - - - \\
														& vérifications des connexions& + & - & - & ++ & - \\
														& facilité d'extension				& + & - & - & + & - \\
\end{xltabular}

\end{landscape}

%\end{document}



%\end{document}
