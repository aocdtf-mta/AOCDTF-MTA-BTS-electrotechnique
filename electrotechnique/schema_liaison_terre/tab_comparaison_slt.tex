%--------------------------------------
%ELECTROTECHNIQUE - SCHEMA DE LIAISON A LA TERRE
%--------------------------------------

%utiliser les environnement \begin{comment} \end{comment} pour mettre en commentaire le préambule une fois la programmation appelée dans le document maître (!ne pas oublier de mettre en commentaire \end{document}!)

\begin{comment}

\documentclass[a4paper, 11pt, twoside, fleqn]{memoir}

\usepackage{AOCDTF}

\marqueurchapitre
\decoupagechapitre{1} %juste pour éviter les erreurs lors de la compilation des sous-programmations (passera en commentaire)

%lien d'édition des figures Tikz sur le site mathcha.io (rajouter le lien d'une modification effectuée sur la figure tikz avec le nom du modificateur car il n'y a qu'un lien par compte)

%lien éditeur Bruno Douchy : https://www.mathcha.io/editor/zjygnFElSdyhJ72e3zT5ZgqwBT4DKnovswpXn1q

%--------------------------------------
%corps du document
%--------------------------------------

\begin{document} %corps du document
	\openleft %début de chapitre à gauche

\end{comment}

\begin{xltabular}{\linewidth}{X l c c c c c}
\caption{Comparaison des différents schémas de liaison à la terre} \\
\toprule
\multicolumn{2}{c}{\thead{Critères de comparaison}} & \thead{TT} & \thead{TN-S} & \thead{TN-C} & \thead{IT\\individuelles} & \thead{IT\\interconnectées} \\
\midrule
\endfirsthead
\multicolumn{7}{l}{\small\textit{Page précédente}} \\
\midrule
\multicolumn{2}{c}{\thead{Critères de comparaison}} & \thead{TT} & \thead{TN-S} & \thead{TN-C} & \thead{IT\\individuelles} & \thead{IT\\interconnectées} \\
\midrule
\endhead
\midrule
\multicolumn{7}{r}{\small\textit{Page suivante}} \\
\endfoot
\bottomrule
\endlastfoot
\multicolumn{7}{l}{\textit{Protection des personnes contre les chocs électriques}} \\ 	
\middashrule													
& Contacts directs							& + 	& + 	& + 	& + 	& + \\
\addlinespace
\multicolumn{7}{l}{\textit{Protection des biens contre les risques d'incendie ou d'explosion d'origine électrique}} \\
\middashrule													
& Incendie et explosion					&  - 	& - -  	& interdit 	& + 	& - - \\
\addlinespace
\multicolumn{7}{l}{\textit{Continuité de service}} \\
\middashrule
														& Creux de tension 				& + & - & - & ++ & - \\
														& Sélectivité							& - & + & + & ++ & + \\
														& Déclenchement					& - & - & - & + & - \\
														& Temps de recherche			& - & + & + & - & + \\
														& Temps de réparation			& - - & - - - & - - - & - & - - - \\
														\addlinespace
\multicolumn{7}{l}{\textit{Protection contre les surtensions}} \\
\middashrule
		& Foudre sur la HT					& - & + & + & + & + \\
														& Claquage du transformateur	& - & + & + & + & + \\
														\addlinespace
\multicolumn{7}{l}{\textit{Compatibilité électromagnétique}} \\ 		
\middashrule
& Rayonnements 					& + & - & - - & ++ & - \\
														& Chute de tension					&  + & - & - & ++ & - \\
														& Harmoniques						&  + & + & - - & + & + \\
\addlinespace
\multicolumn{7}{l}{\textit{Coût à la conception}} \\
\middashrule
					 		& \'Etude la sélectivité 			& - & + & + & ++ & + \\
														& Calcul de $L_{max}$			& + & - & - & ++ & - \\
\addlinespace
\multicolumn{7}{l}{\textit{Coût à l'installation}}\\
\middashrule
					 		& Nombre de câbles	 			& + & + & ++ & + & + \\
														& Nombre de pôles		 			& + & + & ++ & + & + \\
														& Pose des câbles		 			& - & - - & - - & ++ & - - \\
														& Matériel spécifiques	 		& - & + & + & - & + \\
\noalign{\break} %impose le saut de page au tableau tout en répartissant verticalement le tableau
\multicolumn{7}{l}{\textit{Coût à l'exploitation}} \\
\middashrule
							& Recherche de défauts			& - & + & + & - - & + \\
														& Coûts des réparations			& - - & - - - & - - - & - &  - - - \\
														& Vérifications des connexions& + & - & - & ++ & - \\
														& Facilité d'extension				& + & - & - & + & - \\
\end{xltabular}
%\end{document}

