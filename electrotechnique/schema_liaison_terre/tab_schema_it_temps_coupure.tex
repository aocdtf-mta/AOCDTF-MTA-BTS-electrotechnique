%--------------------------------------
%ELECTROTECHNIQUE - SCHEMA DE LIAISON A LA TERRE
%--------------------------------------

%utiliser les environnement \begin{comment} \end{comment} pour mettre en commentaire le préambule une fois la programmation appelée dans le document maître (!ne pas oublier de mettre en commentaire \end{document}!)

\begin{comment}

\documentclass[a4paper, 11pt, twoside, fleqn]{memoir}

\usepackage{AOCDTF}

\marqueurchapitre
\decoupagechapitre{1} %juste pour éviter les erreurs lors de la compilation des sous-programmations (passera en commentaire)

%lien d'édition des figures Tikz sur le site mathcha.io (rajouter le lien d'une modification effectuée sur la figure tikz avec le nom du modificateur car il n'y a qu'un lien par compte)

%lien éditeur Bruno Douchy : https://www.mathcha.io/editor/zjygnFElSdyhJ72e3zT5ZgqwBT4DKnovswpXn1q

%--------------------------------------
%corps du document
%--------------------------------------

\begin{document} %corps du document
	\openleft %début de chapitre à gauche

\end{comment}

\begin{table}[H]
\caption{Temps de coupure maximal des disjoncteurs en schéma IT\label{tab:schema_it_temps_coupure}}
\begin{threeparttable} %note dans tableau
\begin{tabularx}{\textwidth}{C C C}
\toprule
\multirow[c]{2}{*}{\thead{Réseaux usuels}} & \multicolumn{2}{c}{\thead{Temps de coupure maximal (\si{\milli\second})}}\\
\cmidrule(lr){2-3} 
	& $U_{L}=\SI{50}{\volt}$ 	& 			$U_{L}=\SI{25}{\volt}$  \\
\midrule
\multicolumn{3}{l}{\textit{Neutre non distribué}} \\
\middashrule
\SI{127}{\volt}/\SI{230}{\volt}		& 800		& 400 \\
\SI{230}{\volt}/\SI{400}{\volt}		& 400		& 200 \\
\SI{400}{\volt}/\SI{690}{\volt}		& 200		& 60 \\
\SI{690}{\volt}/\SI{1000}{\volt}	& 100		& 20 \\
\addlinespace
\multicolumn{3}{l}{\textit{Neutre distribué\tnote{1}}} \\
\middashrule
\SI{127}{\volt}/\SI{230}{\volt}		& 5000		& 1000 \\
\SI{230}{\volt}/\SI{400}{\volt}		& 800		& 500 \\
\SI{400}{\volt}/\SI{690}{\volt}		& 400		& 200 \\
\SI{690}{\volt}/\SI{1000}{\volt}	& 200		& 80 \\
\bottomrule 
\end{tabularx}
\begin{tablenotes}
    \item[1] les installations monophasées sont considérées comme des installations à neutre distribué.
\end{tablenotes}
\end{threeparttable} %note dans tableau
\end{table}

%\end{document}

