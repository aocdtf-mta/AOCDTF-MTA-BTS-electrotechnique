%--------------------------------------
%ELECTROTECHNIQUE - SCHEMA DE LIAISON A LA TERRE
%--------------------------------------

%utiliser les environnement \begin{comment} \end{comment} pour mettre en commentaire le préambule une fois la programmation appelée dans le document maître (!ne pas oublier de mettre en commentaire \end{document}!)

\begin{comment}

\documentclass[a4paper, 11pt, twoside, fleqn]{memoir}

\usepackage{AOCDTF}

\marqueurchapitre
\decoupagechapitre{1} %juste pour éviter les erreurs lors de la compilation des sous-programmations (passera en commentaire)

%lien d'édition des figures Tikz sur le site mathcha.io (rajouter le lien d'une modification effectuée sur la figure tikz avec le nom du modificateur car il n'y a qu'un lien par compte)

%lien mathcha Nom Prénom : 


%--------------------------------------
%corps du document
%--------------------------------------

\begin{document} %corps du document
	\openleft %début de chapitre à gauche

\end{comment}

\begin{figure}[H]
\caption{Facteur de correction m à appliquer aux abaques des longueurs maximales des câbles $L_{max}$}
\begin{tabularx}{\linewidth}{c X c c c c}
\toprule
\multirow[c]{2}{*}{\thead{Circuit}}		& \makecell[C]{\multirow[c]{2}{*}{\thead{Matériau conducteur}}}			& \multicolumn{4}{c}{\thead{$m=S_{ph}/S_{PE(N)}$}}\\
\cmidrule(lr) {3-6} 
	&	&	$m=1$ &	$m=2$	&	$m=3$	& $m=4$ \\
\midrule
\multirow[c]{2}{*}{3P + N ou P + N}	& cuivre		& 1 		& 0,67		&	0,50	& 0,40 \\
															& aluminium	& 0,62 	& 0,42		&	0,31	& 0,25 \\
\bottomrule
\end{tabularx}
\end{figure}

%\end{document}

