%--------------------------------------
%chap_2_chimie_conduction_electrique
%--------------------------------------

\begin{comment}

\documentclass[a4paper, 11pt, twoside, fleqn]{memoir}

\usepackage{AOCDTF}

%--------------------------------------
%CANEVAS
%--------------------------------------

\newcommand\BoxColor{\ifcase\thechapshift blue!30\or brown!30\or pink!30\or cyan!30\or green!30\or teal!30\or purple!30\or red!30\or olive!30\or orange!30\or lime!30\or gray!\or magenta!30\else yellow!30\fi} %définition de la couleur des marqueurs de chapitre

\newcounter{chapshift} %compteur de chapitre du marqueur de chapitre
\addtocounter{chapshift}{-1}
	
\newif\ifFrame %instruction conditionnelle pour les couleurs des pages
\Frametrue

\pagestyle{plain}

% the main command; the mandatory argument sets the color of the vertical box
\newcommand\ChapFrame{%
\AddEverypageHook{%
\ifFrame
\ifthenelse{\isodd{\value{page}}}
  {\backgroundsetup{contents={%
  \begin{tikzpicture}[overlay,remember picture]
  \node[
  	rounded corners=3pt,
    fill=\BoxColor,
    inner sep=0pt,
    rectangle,
    text width=1.5cm,
    text height=5.5cm,
    align=center,
    anchor=north west
  ] 
  at ($ (current page.north west) + (-0cm,-2*\thechapshift cm) $) %nombre négatif = espacement des marqueurs entre les différents chapitres (à régler en fin de rédaction) (4.5cm vaut un espacement équivalement à la hauteur du marqueur, une page peut en contenir 6 avec cet espacement-la mais il est le plus équilibré)
    {\rotatebox{90}{\hspace*{.5cm}%
      \parbox[c][1.2cm][t]{5cm}{%
        \raggedright\textcolor{black}{\sffamily\textbf{\leftmark}}}}};
  \end{tikzpicture}}}
  }
  {\backgroundsetup{contents={%
  \begin{tikzpicture}[overlay,remember picture]
  \node[
  	rounded corners=3pt,
    fill=\BoxColor,
    inner sep=0pt,
    rectangle,
    text width=1.5cm,
    text height=5.5cm,
    align=center,
    anchor=north east
  ] 
  at ($ (current page.north east) + (-0cm,-2*\thechapshift cm) $) %nombre négatif = espacement des marqueurs entre les différents chapitres (à régler en fin de rédaction) (4.5cm vaut un espacement équivalement à la hauteur du marqueur, une page peut en contenir 6 avec cet espacement-la mais il est le plus équilibré)
    {\rotatebox{90}{\hspace*{.5cm}%
      \parbox[c][1.2cm][t]{5cm}{%
        \raggedright\textcolor{black}{\sffamily\textbf{\leftmark}}}}};
  \end{tikzpicture}}}%
  }
  \BgMaterial%
  \fi%
}%
  \stepcounter{chapshift}
}

\renewcommand\chaptermark[1]{\markboth{\thechapter.~#1}{}} %redéfinition du marqueur de chapitre pour ne contenir que le titre du chapitre

%utiliser les environnement \begin{comment} \end{comment} pour mettre en commentaire le préambule une fois la programmation appelée dans le fichier maître

%--------------------------------------
%corps du document
%--------------------------------------

\begin{document} %corps du document
	\openleft %début de chapitre à gauche
	\Frametrue %défini la booléenne Frame comme vrai -> marqueurs de chapitre

\end{comment}


\chapter{Chimie de la conduction électrique}
\ChapFrame %appel du marqueur de chapitre

\section{Dernière couche électronique}

\subsection{\'Electron-volt}

Les couches électroniques $K, L, M\ldots$ répartissent les électrons autour du noyau atomique :
\begin{itemize}
\item Plus un électron est proche du noyau atomique, plus l'énergie nécessaire pour arracher l'électron du champ électrique du noyau sera grande\,;
\item \'Energie quantifiée en \emph{électron-volt} \si{\electronvolt}.
\end{itemize}
Un \emph{électron-volt} est la mesure physique de l'énergie cinétique d'un électron accéléré sous l'action d'une \emph{différence de potentiel} d'$\SI{1}{\volt}$. Il est égal à : 

\begin{formule}{Valeur expérimentale de l'\electronvolt}{valeur_experimentale_electronvolt}
	\begin{align} 
		U &= \frac{W}{Q} \\
		\electronvolt &= \sqrt{\frac{2h\alpha}{\mu\clight}}\frac{W}{Q} \\
		&=\SI{1,602176634e-19}{\joule} \nonumber
	\end{align}

\begin{numvariables}
U						& différence de potentiel					& volt						& \volt								& \volt  				& \si{\kilogram\square\meter\per\cubic\second\per\ampere} \\
W						& énergie											& joule						& \joule								& \joule				& \si{kg.m^{2}/s^{2}} \\
Q						& charge électrique							& coulomb					& \coulomb						& \coulomb		& \si{\ampere\second} \\
\electronvolt 		& électron-volt 									& joule 					& \si\joule 							& \electronvolt	& \SI{1,602176634e-19}{\joule} \\
h 						& constante de Planck 						& joule seconde 		& \si{\joule\second	}			& h					& \SI{6,62607015e-34}{\joule\second} \\
\alpha				& constante de structure fine 				& sans dimension 		&										& \alpha			& \num{7,2973525564e-3} \\
\mu					& perméabilité magnétique du vide	& henry par mètre		& \si{\henry\per\meter}		& \mu				& \SI{4\pi e-7}{\henry\per\meter} \\
\clight				& vitesse de la lumière dans le vide	& mètre par seconde & \si{\meter\per\second}	& \clight			& \SI{2,99792458e8}{\meter\per\second}	
\end{numvariables}
\end{formule}



%\end{document}

