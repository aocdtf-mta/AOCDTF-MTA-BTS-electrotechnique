%utiliser les environnement \begin{comment} \end{comment} pour mettre en commentaire le préambule une fois la programmation appelée dans le document maître (!ne pas oublier de mettre en commentaire \end{document}!)

%\begin{comment}

\documentclass[varwidth=\maxdimen]{standalone}

%--------------------------------------
%préambule
%--------------------------------------

\usepackage{multirow, ragged2e, ltablex, booktabs, makecell, arydshln, cellspace, tabu} %gestion fine des tableaux
\usepackage{pgf, tikz, tikz-qtree, pgfplots, pgfplotstable} %création de figures et schémas
\usepackage{chemfig, bohr, tikzorbital, chemgreek, expl3, xparse, l3keys2e, xargs, verbatim} %gestion de l'écriture en chimie
\usepackage{modiagram} %orbitale atomique
\usepackage[export]{adjustbox}

%--------------------------------------
%corps du document
%--------------------------------------

\begin{document} %corps du document

%\end{comment}

\begin{tabular}{c c c c c c}

\thead{Représentation en cases quantiques :}
&
\adjustbox{valign=t}{ %alignement des figures avec le haut de la cellule
	\begin{modiagram}[style=square, labels]
     	\AO(0cm){s}[label={$1s^2$}]{0}
	\end{modiagram}}
& 
\adjustbox{valign=t}{ %alignement des figures avec le haut de la cellule
	\begin{modiagram}[style=square, labels]
     	\AO(0cm){s}[label={$2s^2$}]{0}
	\end{modiagram}}
&
\adjustbox{valign=t}{ %alignement des figures avec le haut de la cellule
	\begin{modiagram}[style=square, labels]
        \AO(0cm){s}{0}
        \AO(0,7cm){s}[label={$2p^6$}]{0}
        \AO(1,4cm){s}{0}
	\end{modiagram}}
&
\adjustbox{valign=t}{ %alignement des figures avec le haut de la cellule
	\begin{modiagram}[style=square, labels]
     	\AO(0cm){s}[label={$3s^2$}]{0}
	\end{modiagram}}
&
\adjustbox{valign=t}{ %alignement des figures avec le haut de la cellule
	\begin{modiagram}[style=square, labels]
        \AO(0cm){s}{0;up}
        \AO(0,7cm){s}[label={$3p^1$}]{0;}
        \AO(1,4cm){s}{0;}
	\end{modiagram}} \\
\end{tabular}


\end{document}
