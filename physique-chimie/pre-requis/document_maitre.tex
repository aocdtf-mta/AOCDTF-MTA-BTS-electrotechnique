%--------------------------------------
%appel de la classe de document et de ses options
%--------------------------------------

\documentclass[a4paper, 11pt, twoside, fleqn]{memoir}

\usepackage{AOCDTF}

\addbibresource{../../bibliographies/chimie.bib} %à redéfinir une fois le cours entamé le chemin ne sera pas le même selon les cours !
\addbibresource{../../bibliographies/sciences_naturelles.bib} %à redéfinir une fois le cours entamé le chemin ne sera pas le même selon les cours !
\addbibresource{../../bibliographies/electrotechnique.bib} %à redéfinir une fois le cours entamé le chemin ne sera pas le même selon les cours !

%--------------------------------------
%données du document
%--------------------------------------

\corpsdemetier{Métiers des technologies associées}
\acronymemetier{MTA}

\formation{BTS \'Electrotechnique}
\matiere{Physique-chimie}
\cours{Pré-requis}

\auteura{Bruno}{Douchy}
\acronymemetierauteura{MTA}
\auteurb{}{}
\acronymemetierauteurb{nul}
\auteurc{}{}
\acronymemetierauteurc{nul}
\auteurd{}{}
\acronymemetierauteurd{nul}

\decoupagechapitre{5} %espacement des marqueurs entre les différents chapitres (à régler en fin de rédaction) (5cm vaut un espacement équivalement à la hauteur du marqueur, une page ne peut en contenir que 5 avec cet espacement-la mais il est le plus équilibré)

%--------------------------------------
%corps du document
%--------------------------------------

\begin{document} %corps du document

%--------------------------------------
%préface, page de couverture, table des matières...
%--------------------------------------

\frontmatter

\Framefalse %défini la booléenne Frame comme faux

	%--------------------------------------
	%page de couverture et de titre
	%--------------------------------------

\pagecouverture
\pagetitre
\marqueurchapitre

	\pagestyle{plain} %style de page avec en-tête et pied-de-page
	\pagenumbering{roman}
	\openany
	
	%--------------------------------------
	%listes de contenus
	%--------------------------------------
	
	{\hypersetup{linkcolor=black}\tableofcontents*} %table des matières en noir
	\newpage
	{\hypersetup{linkcolor=black}\listoftables*} %liste des tableaux en noir
	\newpage
	{\hypersetup{linkcolor=black}\listoffigures*} %liste des figures en noir
	{\hypersetup{linkcolor=black}\listtheoremname\listtheorems{formule}} %liste des équations en noir
	{\hypersetup{linkcolor=black}\listdefinitionname\listtheorems{definition}} %liste des définitions en noir
	{\hypersetup{linkcolor=black}\listexemplename\listtheorems{exemple}} %liste des définitions en noir

	\openright %début de chapitre à "droite" mais comme demarrage de la numérotation inversé avec la page de titre, ça décale l'ouverture des chapitre à gauche

	%--------------------------------------
	%inclusion du chapitre d'introduction
	%--------------------------------------
	
	\input{chap_preface}
	
%--------------------------------------
%corps de texte, annexes
%--------------------------------------

\mainmatter

\Frametrue %défini la booléenne Frame comme vrai -> marqueurs de chapitre
	
	%--------------------------------------
	%inclusion des chapitres
	%--------------------------------------

	\include{chap_chimie_atomique}
	\include{chap_chimie_conduction_electrique}

	%--------------------------------------
	%style des annexes
	%--------------------------------------

	\Framefalse %défini la booléenne Frame comme false -> pas de marqueurs de chapitre
	\appendix %appel des annexes
	\appendixpage

	%--------------------------------------
	%inclusion des chapitres
	%--------------------------------------

	\include{ann_addendum_chimie_atomique}
	\include{ann_unites_symboles}
	
%--------------------------------------
%conclusion, bibliographie
%--------------------------------------

\backmatter
	
	%--------------------------------------
	%inclusion des chapitres
	%--------------------------------------
	
	
	
	%--------------------------------------
	%bibliographie
	%--------------------------------------

	\nocite{Coarer2003} %appel les références utilisées pour la construction du document mais non citées (clés insensibles à la casse)
	
	\printbibliography %ajout des références bibliographiques

\end{document}
