%utiliser les environnement \begin{comment} \end{comment} pour mettre en commentaire le préambule une fois la programmation appelée dans le document maître (!ne pas oublier de mettre en commentaire \end{document}!)

%--------------------------------------
%chap_0_preface
%--------------------------------------

\begin{comment}

\documentclass[a4paper, 11pt, twoside]{memoir}

\usepackage{AOCDTF}

%--------------------------------------
%CANEVAS
%--------------------------------------

\newcommand\BoxColor{\ifcase\thechapshift blue!30\or brown!30\or pink!30\or cyan!30\or green!30\or teal!30\or purple!30\or red!30\or olive!30\or orange!30\or lime!30\or gray!\or magenta!30\else yellow!30\fi} %définition de la couleur des marqueurs de chapitre

\newcounter{chapshift} %compteur de chapitre du marqueur de chapitre
\addtocounter{chapshift}{-1}
	
\newif\ifFrame %instruction conditionnelle pour les couleurs des pages
\Frametrue

\pagestyle{plain}

% the main command; the mandatory argument sets the color of the vertical box
\newcommand\ChapFrame{%
\AddEverypageHook{%
\ifFrame
\ifthenelse{\isodd{\value{page}}}
  {\backgroundsetup{contents={%
  \begin{tikzpicture}[overlay,remember picture]
  \node[
  	rounded corners=3pt,
    fill=\BoxColor,
    inner sep=0pt,
    rectangle,
    text width=1.5cm,
    text height=5.5cm,
    align=center,
    anchor=north west
  ] 
  at ($ (current page.north west) + (-0cm,-2*\thechapshift cm) $) %nombre négatif = espacement des marqueurs entre les différents chapitres (à régler en fin de rédaction) (4.5cm vaut un espacement équivalement à la hauteur du marqueur, une page peut en contenir 6 avec cet espacement-la mais il est le plus équilibré)
    {\rotatebox{90}{\hspace*{.5cm}%
      \parbox[c][1.2cm][t]{5cm}{%
        \raggedright\textcolor{black}{\sffamily\textbf{\leftmark}}}}};
  \end{tikzpicture}}}
  }
  {\backgroundsetup{contents={%
  \begin{tikzpicture}[overlay,remember picture]
  \node[
  	rounded corners=3pt,
    fill=\BoxColor,
    inner sep=0pt,
    rectangle,
    text width=1.5cm,
    text height=5.5cm,
    align=center,
    anchor=north east
  ] 
  at ($ (current page.north east) + (-0cm,-2*\thechapshift cm) $) %nombre négatif = espacement des marqueurs entre les différents chapitres (à régler en fin de rédaction) (4.5cm vaut un espacement équivalement à la hauteur du marqueur, une page peut en contenir 6 avec cet espacement-la mais il est le plus équilibré)
    {\rotatebox{90}{\hspace*{.5cm}%
      \parbox[c][1.2cm][t]{5cm}{%
        \raggedright\textcolor{black}{\sffamily\textbf{\leftmark}}}}};
  \end{tikzpicture}}}%
  }
  \BgMaterial%
  \fi%
}%
  \stepcounter{chapshift}
}

\renewcommand\chaptermark[1]{\markboth{\thechapter.~#1}{}} %redéfinition du marqueur de chapitre pour ne contenir que le titre du chapitre %à personnaliser selon le nombre de chapitre dans le cours

%--------------------------------------
%corps du document
%--------------------------------------

\begin{document} %corps du document
	\openleft %début de chapitre à gauche

\end{comment}

\chapter{Préface}

\section{Avant-propos}

Ces cours de physique -- chimie appliquées à l'électrotechnique sont à la disposition des étudiants BTS \'Electrotechnique pour appréhender les origines de l'énergie électrique et sa maitrise appliquée aux environnements professionnels. La compréhension de ces matières est indispensable pour saisir la nature un peu abstraite de l'énergie électrique que les détenteurs du BTS \'Electrotechnique se doivent de maitriser.\\
Ces cours sont la référence pédagogique pour toute la formation et ils regroupent toutes les connaissances théoriques à maitriser à l'issue de la formation. Il couvrira donc le programme initial et abordera également des notions de physique -- chimie supplémentaires, pour favoriser à l'étudiant sortant lors de son insertion professionnelle. Chaque fin de chapitre est composé d'un résumé de l'essentiel à retenir.\\

L'énergie électrique puise ses origines dans l'aspect chimique de la matière. Ce cours abordera donc des notions élémentaires de chimie atomique et quantique, qui sont des chapitres nécessaires pour comprendre le comportement des électrons, pourquoi tel ou tel élément est conducteur ou encore ce qu'on appelle un semi-conducteur.\\Tout ce qui sera abordé dans ce chapitre aura des répercussions dans les chapitres suivants de génie électrique et de physique -- chimie. Toutefois, comme les notions présentées pour ce cours de pré-requis de chimie sont issues de références de sciences techniques de l'ingénieur, leur maîtrise ne sera donc pas à exigée. Néanmoins, leur compréhension permettra de préparer judicieusement le terrain d'une éventuelle poursuite des études dans le secteur de l'électrotechnique.


\section{Références additionnelles et \textrm{\LaTeX}}

Ces cours n'ayant aucune prétention d'exhaustivité ni de contenir des savoirs innés, on retrouve en fin de document une bibliographie interactive compilant les références (livres, cours, sites internet\ldots) dont sont issues les informations contenues dans ce cours.\\Les liens de couleurs ~\textcolor{Blue}{\rule{1.5em}{1.2ex}}~ vous permettront d'accéder à ces documents contenus dans une base de données sur l'Intranet pour ceux en libre accès, ou de vous rediriger vers une libraire en ligne pour les quelques livres de références que nous vous conseillons d'investir.\\

La rédaction de ces documents est réalisée à partir d'un environnement de programmation \LaTeX. C'est un outil utilisé pour rédiger les livres, les publications scientifiques et universitaires ou encore les thèses de doctorats de façon normée.\\Il s'agit donc d'introduire cette nouvelle matière qui est enseignée dans les études supérieures des sciences techniques. Elle vous permettra autant de se familiariser avec un univers de programmation et ses mécanismes que de maitriser un outil de rédaction de document de qualité. Cela apportera un gain de crédibilité certain sortants lors de la rédaction de documents tels que le dossier du travail de réception ou les documents professionnels comme les lettres et les présentations.


%\end{document}